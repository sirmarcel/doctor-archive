%!TEX root = ../da.tex

\usepackage[a-2b]{pdfx}

\usepackage[utf8]{inputenc}
% \usepackage[german,english]{babel}
\usepackage[british]{babel}
\usepackage[T1]{fontenc}

% Fonts
\usepackage{amsfonts}
\usepackage[fleqn]{amsmath}
\usepackage{bm}
\usepackage{bbold}

% Grafix
\usepackage{graphicx}

% Color definitions for drawings etc.
% The main accent color is just red,
% extra_i should be easily distinguishable
% other colors
\usepackage{xcolor}

\definecolor{extra_0}{HTML}{66c2a5}
\definecolor{extra_1}{HTML}{ffd92f}
\definecolor{extra_2}{HTML}{8da0cb}
\definecolor{extra_3}{HTML}{e78ac3}
\definecolor{extra_4}{HTML}{a6d854}
\definecolor{extra_5}{HTML}{fc8d62}

\definecolor{accent}{HTML}{AA0022}
\definecolor{dark_accent}{HTML}{500310}
\definecolor{accent_2}{HTML}{006C66}

\definecolor{teal}{HTML}{007766}
\definecolor{orange}{HTML}{EE7733}

% Figures
\usepackage[lofdepth,lotdepth]{subfig}
\usepackage{float}
%\usepackage{caption} % disabled for tufte
\usepackage{epstopdf}


% Tables
% \usepackage{longtable}
% \usepackage{arydshln}
\usepackage{booktabs}
\usepackage{multirow}

% The fancyvrb package lets us customize the formatting of verbatim
% environments.  We use a slightly smaller font.
\usepackage{fancyvrb}
\fvset{fontsize=\normalsize}

% Pretty physics things
\usepackage{braket}
\usepackage[separate-uncertainty=true]{siunitx}
\usepackage{mathtools}
\usepackage{chemformula}    % chemical formulas
\usepackage[makeroom]{cancel}

% Pretty code things
% \usepackage{minted}
%TC:group minted 0 0

% Referencing
\usepackage{url}
% package is already loaded by pdftex
\hypersetup{
  colorlinks=true,
  linkcolor=dark_accent,
  anchorcolor=black,
  citecolor=dark_accent,
  urlcolor=black,
  bookmarksopenlevel=chapter,
  % ignored, because pdftex pretends to set it, but doesn't
  % pdfkeywords={machine learning, quantum mechanics, materials science, automatic differentiation, thermal transport},
  % required by TUB
  pdflang={en-GB},
  pdfdisplaydoctitle=true,
  pdfpagelayout=OneColumn,
}
\usepackage[noabbrev]{cleveref}
\usepackage{csquotes}
\usepackage[backend=biber,style=phys,biblabel=brackets,backref=true,eprint=true,articletitle=true,pageranges=false]{biblatex}

\usepackage[toc]{appendix}

\usepackage{enumitem}
\setlist[description]{%
  font={\upshape\bfseries\ttfamily} % set the label font
%  font={\bfseries\sffamily\color{red}}, % if colour is needed
}

% Drawing
\usepackage{tikz}
\usetikzlibrary[arrows.meta,bending]
\usetikzlibrary{positioning}

% Pretty type
% \usepackage{underscore} % we do not need to escape _ in text mode
\usepackage[super]{nth}
\usepackage{nicefrac}
\usepackage{scalefnt}
\usepackage[normalem]{ulem}  % strikeout but no underlined emphasis
\usepackage{lettrine}
% \usepackage[babel,final]{microtype} % disabled for tufte
% Prints a trailing space in a smart way.
\usepackage{xspace}
\usepackage{relsize}
\usepackage{hyphenat}
\hyphenation{Schrö-ding-er}
\hyphenation{well-doc-u-men-ted}
\hyphenation{app-roach-es}
\hyphenation{da-ta-set}
\hyphenation{da-ta-sets}
\hyphenation{struc-tu-re-pro-per-ty}
\hyphenation{equi-var-iant}
\hyphenation{quan-ti-fy}
\hyphenation{atom-ic}
\hyphenation{Gauss-ian}
\hyphenation{e-qui-li-bri-um}
\hyphenation{first-prin-cip-les}
\hyphenation{an-har-mo-ni-ci-ty}
\hyphenation{prin-cip-les}
\hyphenation{ther-mal}
% \renewcommand*\ttdefault{cmvtt} % use Computer Modern Typewriter Proportional for tt, which is Latin Modern Mono Proportional irl
% fancy fixes for the fake palatino we're using -- the order matters
\usepackage[tighter,largesc]{newpxtext}
\usepackage[smallerops]{newpxmath}


%!TEX root = da.tex

% deprecated
\newcommand{\abr}[1]{\textsc{#1}}
% \newcommand{\mlp}{\abr{mlp}}
% \newcommand{\mlps}{\abr{mlp}s}
% \newcommand{\mpnn}{\abr{mpnn}}
% \newcommand{\mpnns}{\abr{mpnn}s}
% \newcommand{\hfacf}{\abr{hfacf}}



% General
\newcommand{\scare}[1]{\lq#1\rq} 
\newcommand{\newterm}[1]{\emph{#1}}
\newcommand{\etal}{et~al.}
\newcommand{\el}{{\,\ldots}\xspace}
\newcommand{\yes}{\checkmark}
\newcommand{\no}{$\times$}
\newcommand{\quotes}[1]{``#1''}

\newcommand{\symset}[1]{\mathcal{#1}}  % symbol of a set
% \newcommand{\curlyset}[2]{\{\, #1\, |\, #2 \, \}}
\newcommand{\curlyset}[2]{\Set{#1 | #2}}


% Shorthand

\newcommand{\zro}{\ch{ZrO2}}

\newcommand{\dsgdb}{\texttt{qm9}\xspace}
\newcommand{\dsba}{\texttt{ba10}\xspace}
\newcommand{\dstco}{\texttt{nmd18}\xspace} % was: tco3; Christopher Sutton prefers NMD18
\newcommand{\dstcou}{\texttt{nmd18u}\xspace}
\newcommand{\dstcor}{\texttt{nmd18r}\xspace}



% GK
\newcommand{\tk}{\boldsymbol{\kappa}}
\newcommand{\kt}{\kappa(\tau)}
\newcommand{\ehfacf}{C}
\newcommand{\ehfacft}{\boldsymbol{C}}
\newcommand{\tc}{t_{\text{c}}}
\newcommand{\nhf}{n_{\text{hf}}}
\newcommand{\omegaf}{\omega_{\text{filter}}}
\newcommand{\keps}{\epsilon_{\text{c}}}
\newcommand{\coco}[2]{(#1, \qty{#2}{ns})}
\newcommand{\zrocc}{\coco{1500}{1}}
\newcommand{\zrouc}{\coco{96}{0.1}}
\newcommand{\zrosc}{\coco{768}{0.5}}
\newcommand{\snsecc}{\coco{864}{2}}

% ... J
\newcommand{\J}{\boldsymbol{J}}
\newcommand{\Jfull}{\boldsymbol{J}_\text{full}}
\newcommand{\Jgauge}{\boldsymbol{P}}
\newcommand{\Jint}{\boldsymbol{J}_\text{int}}
\newcommand{\Jdiff}{\boldsymbol{J}_\text{disp}}
\newcommand{\Jpot}{\boldsymbol{J}_\text{pot}}
\newcommand{\Jconv}{\boldsymbol{J}_\text{conv}}
\newcommand{\Jintrt}{\boldsymbol{J}_{\text{int}}^{\boldsymbol{r}(t)}}
\newcommand{\Jmpnn}{\J_{\text{Hardy}}^{\text{MPNN}}}
\newcommand{\Jglp}{\J_{\text{Hardy}}^{\text{GLP}}}
\newcommand{\Jdft}{\J_{\text{virial}}}
\newcommand{\Jfan}{\J_{\text{Fan}}}
\newcommand{\Junfolded}{\Jhardy^{\text{unfolded}}}
\newcommand{\Jhardy}{\J_{\text{Hardy}}}
\newcommand{\Jvirial}{\J_{\text{virial}}}

% the next generation
\newcommand{\Junf}{\J^{\text{unfolded}}_\text{pot}}
\newcommand{\Jmic}{\J^{\text{mic}}_\text{pot}}
\newcommand{\Jloc}{\J^{\text{local}}_\text{pot}}
\newcommand{\Jsl}{\J^{\text{semi-local}}_\text{pot}}
\newcommand{\Jedg}{\J^{\text{edges}}_\text{pot}}

% ... without the pot suffix
\newcommand{\Jgunf}{\J^{\text{unfolded}}}
\newcommand{\Jgmic}{\J^{\text{mic}}}
\newcommand{\Jgloc}{\J^{\text{local}}}
\newcommand{\Jgsl}{\J^{\text{semi-local}}}
\newcommand{\Jgedg}{\J^{\text{edges}}}

% ... densities
\newcommand{\densj}{\boldsymbol{j}}
\newcommand{\dense}{e}
\newcommand{\densgauge}{\boldsymbol{p}}
\newcommand{\localf}[1]{\Delta(#1)}
\newcommand{\bondf}[2]{\Lambda_{#1}(#2)}


% MD
\newcommand{\phasp}[2]{\Gamma_{#1}^{#2}} % phase-space point
\newcommand{\fullphasp}{\boldsymbol{\Gamma}} % phase-space
\newcommand{\ensemble}{\mathcal{E}}
\newcommand{\td}{t_0}
\newcommand{\Td}{T_0}
\newcommand{\nens}{n}
\newcommand{\Hamiltonian}{\mathcal{H}}
\newcommand{\dt}{\Delta t}

% Graphs/MPNN
\newcommand{\dur}[3]{\frac{\partial U_{#1}}{\partial \R_{#2#3}}}
\newcommand{\durm}[3]{\frac{\partial U_{#1}}{\partial \Rm_{#2#3}}}
\newcommand{\indur}[3]{\partial U_{#1} / \partial \R_{#2#3}}
\newcommand{\indurm}[3]{\partial U_{#1} / \partial \Rm_{#2#3}}

\newcommand{\state}[2]{\boldsymbol{s}_{#1}^{#2}}
\newcommand{\msg}[2]{\boldsymbol{m}_{#1}^{#2}}
\newcommand{\nbh}[1]{\symset{N}(#1)}
\newcommand{\nbht}[1]{\symset{N}^\interactions\!(#1)}
\newcommand{\msgf}[1]{\boldsymbol{M}_{#1}}
\newcommand{\updatef}[1]{\boldsymbol{F}_{#1}}
\newcommand{\edge}[2]{\boldsymbol{r}_{#1#2}}

\newcommand{\interactions}{M}
\newcommand{\cutoff}{r_{\text{c}}}
\newcommand{\onset}{r_{\text{o}}}
\newcommand{\effcutoff}{\cutoff^{\text{eff}}}
\newcommand{\maxcutoff}{\cutoff^{\text{max}}}
\newcommand{\skin}{r_{\text{tol}}}
\newcommand{\fcutoff}{f_{\text{c}}}

\newcommand{\graph}{\mathcal{G}}
\newcommand{\graphunf}{\mathcal{G}^{\text{unf}}}
\newcommand{\vertices}{\mathcal{V}}
\newcommand{\edges}{\mathcal{E}}
\newcommand{\edgest}[1]{\mathcal{E}^\interactions\!(#1)}

\newcommand{\patch}[1]{\mathcal{P}(#1)}

% Statistics
\newcommand{\sampled}{\,\sim\,}

% ML/KRR

% \newcommand{\targetf}{f}
% \newcommand{\approxf}{f}
% \newcommand{\krrweight}{\alpha}
\newcommand{\inspace}{\symset{V}}
\newcommand{\outspace}{\symset{O}}
\newcommand{\featsp}{\symset{H}}

% Math + Physics general
\renewcommand{\vec}[1]{\boldsymbol{#1}}
\newcommand{\grad}{\vec{\nabla}}
\newcommand{\mat}[1]{\boldsymbol{#1}}
\newcommand{\transpose}{\top}
\newcommand{\defvec}[2]{\left[\, #1\, |\, #2 \, \right]^\transpose}
\newcommand{\defvect}[2]{\left[\, #1\, |\, #2 \, \right]}

\newcommand{\op}[1]{\hat #1}
\newcommand{\opr}{\hat{\boldsymbol{r}}}
\newcommand{\opre}{\hat{\boldsymbol{r}}^{\text{e}}}
\newcommand{\opp}{\hat{\boldsymbol{p}}}
\newcommand{\oppe}{\hat{\boldsymbol{p}}^{\text{e}}}
\newcommand{\mel}{m^\text{e}}

\newcommand{\R}{\boldsymbol{r}}
\newcommand{\Rr}{\boldsymbol{r}^0}
\newcommand{\Rnorm}{\overline{\boldsymbol{{r}}}}
\newcommand{\Rrm}{\boldsymbol{r}^{0,\text{mic}}}
\newcommand{\Rm}{\boldsymbol{r}^{\text{mic}}}
\newcommand{\Rconst}{\boldsymbol{r}^{\text{const}}}

% \newcommand{\Re}{\boldsymbol{r}^\text{e}}  % electron position

\newcommand{\Rf}{\boldsymbol{x}}
\newcommand{\Rfc}{x}

\newcommand{\aprop}{\boldsymbol{P}}

\newcommand{\U}{\boldsymbol{u}}
\newcommand{\V}{\boldsymbol{v}}
\newcommand{\momentum}{\boldsymbol{p}}
\newcommand{\F}{\boldsymbol{F}}
\newcommand{\Bary}{\boldsymbol{B}}
\newcommand{\intd}{\text{d}}
\newcommand{\B}[1]{\mathbf{#1}}
\newcommand{\integral}[1]{\int_{#1} \, \text{d}}
\newcommand{\limitintegral}[2]{\int_{#1}^{#2} \, \text{d}}
\newcommand{\dirac}[1]{\delta(#1)}
\newcommand{\defas}{\coloneqq}  % needs \usepackage{mathtools}
\newcommand{\defdas}{\eqqcolon}
\newcommand{\mbeq}{\overset{!}{=}}



\newcommand{\totaldiff}[1]{\frac{\intd}{\intd #1}}
\newcommand{\totaldifff}[2]{\frac{\intd #1}{\intd #2}}
\newcommand{\partdiff}[1]{\frac{\partial}{\partial #1}}
\newcommand{\partdifff}[2]{\frac{\partial #1}{\partial #2}}
\newcommand{\wholes}{{\bf Z}}
\newcommand{\reals}{{\bf R}}
\newcommand{\matone}{{\bf 1}}

\newcommand{\zerovec}{\boldsymbol{0}}

\newcommand{\bigo}[1]{O(#1)}

\newcommand{\magnitude}[1]{|#1|}

\newcommand{\trace}[1]{\text{tr}(#1)}

% Quantum Chemistry
\newcommand{\Hel}{\op{H}_{\text{el}}}

% Solid State
\newcommand{\stress}{\boldsymbol{\sigma}}
\newcommand{\stressc}{\sigma}
\newcommand{\strain}{\boldsymbol{\epsilon}}
\newcommand{\strainc}{\epsilon}

\newcommand{\virial}{\boldsymbol{\Omega}}

% Groups and stuff
\newcommand{\euc}{\text{E}}
\newcommand{\sog}{\text{SO}}
\newcommand{\og}{\text{O}}


% describing atomistic systems
\newcommand{\atom}{\mathcal{A}}
\newcommand{\system}{\mathcal{M}}
\newcommand{\train}{\mathcal{D}_{\text{train}}}
\newcommand{\test}{\mathcal{D}_{\text{test}}}
\newcommand{\ntrain}{n}

\newcommand{\basis}{\boldsymbol{b}}
\newcommand{\offset}{\boldsymbol{n}}
\newcommand{\recip}{\boldsymbol{a}}
\newcommand{\recipn}{\overline{\bm{a}}}

\newcommand{\Rgen}{\symset{R}}
\newcommand{\Rsc}{\symset{R_{\text{sc}}}}
\newcommand{\Rrep}{\symset{R_{\text{rep}}}}
\newcommand{\Rall}{\symset{R_{\text{all}}}}
\newcommand{\Runf}{\symset{R_{\text{unf}}}}
\newcommand{\Basis}{\symset{B}}
\newcommand{\Recip}{\symset{A}}
\newcommand{\Charges}{\symset{Z}}
\newcommand{\Properties}{\symset{P}}

\newcommand{\Momenta}{\symset{P}}

\newcommand{\nsurf}{\boldsymbol{a}}

\newcommand{\anha}{\sigma^{\text{A}}}

% Reps
\newcommand{\rep}{R}

% \newcommand{\MM}{\boldsymbol{M}}
% \newcommand{\M}{\boldsymbol{m}}

\DeclareMathOperator*{\argmin}{arg\,min} % thin space, limits underneath in displays
% \DeclareMathOperator*{\argmin}{argmin} % no space, limits underneath in displays

% URLS
\newcommand{\nicelink}[1]{\href{#1}{\texttt{#1}}}
\newcommand{\nicedoi}[1]{\href{https://doi.org/#1}{\texttt{doi:#1}}}

% Papers
\newcommand{\PaperReps}{
``Representations of molecules and materials for interpolation of\\ quantum-mechanical simulations via machine learning,''\\
\textit{by} Marcel F. Langer, Alex Goeßmann, and Matthias Rupp\\
\textit{in} npj Computational Materials \textbf{8}, 41 (2022)\\
\href{https://doi.org/10.1038/s41524-022-00721-x}{\texttt{doi:10.1038/s41524-022-00721-x}}\\
Referenced as \cite{lgr2022q}.}

\newcommand{\PaperHF}{
``Heat flux for semilocal machine-learning potentials,''\\
\textit{by} Marcel F. Langer, Florian Knoop, Christian Carbogno, Matthias Scheffler, and Matthias Rupp\\
\textit{in} Physical Review B \textit{in press}\\
\href{https://arxiv.org/abs/2303.14434}{\texttt{arXiv:2303.14434}}\\
Referenced as \cite{lksr2023a}.	
}

\newcommand{\PaperGLP}{
``Stress and heat flux via automatic differentiation,''\\
\textit{by} Marcel F. Langer, J. Thorben Frank, and Florian Knoop\\	
\textit{in revision}\\
\href{https://arxiv.org/abs/2305.01401}{\texttt{arXiv:2305.01401}}\\
Referenced as \cite{lfk2023a}.	
}

\usepackage[toc,shortcuts,numberedsection=nameref]{glossaries}

\newacronym{gk}{GK}{Green-Kubo}
\newacronym{ml}{ML}{machine learning}
\newacronym{mic}{MIC}{minimum image convention}
\newacronym{dft}{DFT}{density-functional theory}
\newacronym{he}{HE}{Helfand-Einstein}
\newacronym{aigk}{aiGK}{ab initio Green-Kubo}
\newacronym{md}{MD}{molecular dynamics}
\newacronym{aimd}{aiMD}{ab initio molecular dynamics}
\newacronym{pes}{PES}{potential energy surface}
\newacronym{ad}{AD}{automatic differentiation}
\newacronym{bte}{BTE}{Boltzmann transport equation}
\newacronym{hpc}{HPC}{high-performance computing}
\newacronym{vdos}{VDOS}{vibrational density of states}
\newacronym{hfacf}{HFACF}{heat flux autocorrelation function}
% \newacronym{soap}{SOAP}{Smooth Overlap of Atomic Positions}
\newacronym{mpnn}{MPNN}{message-passing neural network}
\newacronym{glp}{GLP}{graph-based machine-learning potential}
\newacronym{mlp}{MLIP}{machine-learning interatomic potential}
\newacronym{lammps}{LAMMPS}{Large-scale Atomic/Molecular Massively Parallel Simulator}
\newacronym{gga}{GGA}{generalised gradient approximation}
\newacronym{paw}{PAW}{projector-augmented wave}
\newacronym{pbe}{PBE}{Perdew, Burke, and Ernzerhof}
\newacronym{dm21}{DM21}{DeepMind 21}
\newacronym{lda}{LDA}{local-density approximation}
\newacronym{scf}{SCF}{self-consistent field}
\newacronym{fci}{FCI}{full configuration-interaction}
\newacronym{hf}{HF}{Hartree-Fock}
\newacronym{rmsle}{RMSLE}{root mean square logarithmic error}
\newacronym{bo}{BO}{Born-Oppenheimer}
\newacronym{sc}{s.c.}{simulation cell}
\newacronym{jvp}{JVP}{Jacobian-vector product}
\newacronym{vjp}{VJP}{vector-Jacobian product}
\newacronym{rkhs}{RKHS}{reproducing kernel Hilbert space}
\newacronym{rbf}{RBF}{radial basis function}
\newacronym{se}{SE}{squared exponential}
\newacronym{tpu}{TPU}{tensor processing unit}
\newacronym{gpu}{GPU}{graphics processing unit}
\newacronym{sgd}{SGD}{stochastic gradient descent}
\newacronym{adam}{ADAM}{adaptive moment estimation}
\newacronym{iid}{i.i.d.}{independent and identically distributed}
\newacronym{sisso}{SISSO}{sure independence screening and sparsifying operator}
\newacronym{ase}{ASE}{atomic simulation environment}
\newacronym{fcc}{FCC}{face-centred cubic}
\newacronym{bcc}{BCC}{body-centred cubic}
\newacronym{hcp}{HCP}{hexagonal close-packed}
\newacronym{vasp}{VASP}{Vienna Ab initio Simulation Package}

% REPs

\newacronym{ace}{ACE}{atomic cluster expansion}
\newacronym{mace}{MACE}{multi-layer atomic cluster expansion}
\newacronym{bob}{BoB}{bag of bonds}
\newacronym{bs}{BS}{bispectrum}
\newacronym{cm}{CM}{Coulomb matrix}
\newacronym{decaf}{DECAF}{density-encoded canonically-aligned fingerprint}
\newacronym{fchl}{FCHL}{Faber-Christensen-Huang-von Lilienfeld}
\newacronym{gm}{GM}{Gaussian moments}
\newacronym{hdad}{HDAD}{histograms of distances, angles, and dihedral angles}
\newacronym{idmbr}{IDMBR}{inverse-distance many-body representation}
\newacronym{mbtr}{MBTR}{many-body tensor representation}
\newacronym{milad}{MILAD}{moment invariants local atomic descriptors}
\newacronym{mob}{MOB}{molecular orbital basis}
\newacronym{mtp}{MTP}{moment tensor potential}
\newacronym{nice}{NICE}{$N$-body iterative contraction of equivariants}
\newacronym{omf}{OMF}{overlap matrix fingerprint}
\newacronym[plural=SFs, firstplural=symmetry functions]{sf}{SF}{symmetry function}
\newacronym{soap}{SOAP}{smooth overlap of atomic positions}
\newacronym{wst}{WST}{wavelet scattering transform}
\newacronym{snap}{SNAP}{Spectral Neighbor Analysis Potential}
\newacronym{iv}{IV}{internal vectors}
\newacronym{ce}{CE}{cluster expansion}
\newacronym{gap}{GAP}{Gaussian approximation potential}
\newacronym{zora}{ZORA}{scaled zeroeth order regular approximation}
\newacronym{fcp}{FCP}{force constant potential}
\newacronym{gdb17}{GDB-17}{`generated database 17'}
\newacronym{b3lyp}{B3LYP}{Becke 3-parameter Lee-Yang-Parr}

% FFs
\newacronym{ff}{FF}{forcefield}
\newacronym{eam}{EAM}{embedded atom method}



\newacronym{gpr}{GPR}{Gaussian process regression}
\newacronym[plural=HPs]{hp}{HP}{hyperparameter}
\newacronym{krr}{KRR}{kernel ridge regression}
\newacronym{svr}{SVR}{support vector regression}
\newacronym{nn}{NN}{neural network}
\newacronym{mae}{MAE}{mean absolute error}
\newacronym{maxae}{maxAE}{maximum absolute error}
\newacronym{qm}{QM}{quantum mechanics}
\newacronym{rmse}{RMSE}{root mean squared error}
\newacronym{rrmse}{rRMSE}{relative root mean squared error}
\newacronym{rmae}{rMAE}{relative mean absolute error}
\newacronym{ae}{AE}{absolute error}
\newacronym{mse}{RMSE}{mean squared error}

\newacronym{mape}{MAPE}{mean absolute percentage error}
\newacronym{maxape}{maxAPE}{maximum absolute percentage error}

\newacronym{tpe}{TPE}{tree-structured Parzen estimator}

\newacronym{sw}{SW}{Stillinger-Weber}

\newcommand{\mlp}{\gls{mlp}\xspace}
\newcommand{\mlps}{\glspl{mlp}\xspace}
\newcommand{\ff}{\gls{ff}\xspace}
\newcommand{\ffs}{\glspl{ff}\xspace}
\newcommand{\mpnn}{\gls{mpnn}\xspace}
\newcommand{\mpnns}{\glspl{mpnn}\xspace}
\newcommand{\hfacf}{\gls{hfacf}\xspace}
\newcommand{\ad}{\gls{ad}\xspace}
\newcommand{\glp}{\gls{glp}\xspace}
\newcommand{\glps}{\glspl{glp}\xspace}
\newcommand{\mic}{\gls{mic}\xspace}
\newcommand{\md}{\gls{md}\xspace}
\newcommand{\gk}{\gls{gk}\xspace}
\newcommand{\ml}{\gls{ml}\xspace}
\newcommand{\hps}{\glspl{hp}\xspace}
\newcommand{\hp}{\gls{hp}\xspace}
\newcommand{\sfs}{\glspl{sf}\xspace}
% \newcommand{\sf}{\gls{sf}\xspace}
\newcommand{\mbtr}{\gls{mbtr}\xspace}
\newcommand{\soap}{\gls{soap}\xspace}
\newcommand{\ace}{\gls{ace}\xspace}
\newcommand{\krr}{\gls{krr}\xspace}
\newcommand{\nn}{\gls{nn}\xspace}
\newcommand{\nns}{\glspl{nn}\xspace}
\newcommand{\dft}{\gls{dft}\xspace}
\newcommand{\bo}{\gls{bo}\xspace}
\newcommand{\pes}{\gls{pes}\xspace}
\newcommand{\qm}{\gls{qm}\xspace}
\newcommand{\vdos}{\gls{vdos}\xspace}
\newcommand{\aimd}{\gls{aimd}\xspace}
\newcommand{\aigk}{\gls{aigk}\xspace}
\newcommand{\sw}{\gls{sw}\xspace}
\newcommand{\hpc}{\gls{hpc}\xspace}

% code
\newcommand{\software}[1]{\texttt{#1}}
\newcommand{\ase}{\software{ase}\xspace}
\newcommand{\jax}{\software{jax}\xspace}
\newcommand{\glpc}{\software{glp}\xspace}
\newcommand{\aims}{\software{FHI-aims}\xspace}
\newcommand{\vibes}{\software{FHI-vibes}\xspace}
\newcommand{\cmlkit}{\software{cmlkit}\xspace}
\newcommand{\ipi}{\software{i-pi}\xspace}
\newcommand{\schnetpack}{\software{schnetpack}\xspace}
\newcommand{\nequip}{\software{nequip}\xspace}
\newcommand{\mlff}{\software{mlff}\xspace}
\newcommand{\gkx}{\software{gkx}\xspace}
\newcommand{\pytorch}{\software{pytorch}\xspace}

% models
\newcommand{\schnet}{SchNet\xspace}
\newcommand{\sok}{So3krates\xspace}


\def\mathnote#1{%
  \tag*{\rlap{\hspace\marginparsep\smash{\parbox[t]{\marginparwidth}{%
  \footnotesize#1}}}}
}

\makeatletter
\renewcommand{\@chapapp}{}% Not necessary...
\newenvironment{chapquote}[2][2em]
  {\setlength{\@tempdima}{#1}%
   \def\chapquote@author{#2}%
   \parshape 1 \@tempdima \dimexpr\textwidth-1\@tempdima\relax%
   \itshape}
  {\par\bigskip\normalfont\hfill\raggedright--\ \chapquote@author\par \bigskip\bigskip}
\makeatother

\newlist{romanenum}{enumerate}{2}
\setlist[romanenum,1]{label=\textsc{\roman*},align=left,leftmargin=0pt,itemindent=!,labelwidth=0pt,itemsep=0pt,parsep=0.5\baselineskip,widest*=8}
\setlist[romanenum,2]{label=\textsc{\roman{romanenumi}}.\alph*,align=left,leftmargin=3.5ex,labelwidth=*,labelsep*=0pt,itemindent=0pt,itemsep=0pt}
\newlist{romanenum*}{enumerate*}{1}
\setlist[romanenum*,1]{label=(\alph*),afterlabel={{}},ref={\textsc{\roman{romanenumi}}.\alph*},itemjoin={{, }},itemjoin*={{, and }}}


\newcommand{\eref}[1]{reference~\cite{#1}} % explicit reference (singular)
\newcommand{\erefs}[1]{references~\cite{#1}} % explicit reference (plural)
\newcommand{\Erefs}[1]{References~\cite{#1}} % as eref, but always capitalized; use at beginning of sentences
