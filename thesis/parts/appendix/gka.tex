%!TEX root = ../../da.tex

\chapter[Thermal Conductivity with\\ Message-Passing Neural Networks]{Thermal Conductivity with Message-Passing Neural Networks}
\label{ch:si-gka}

\section{Anharmonicity Scores}
\label{sec:si-gka_anha}

To inform the selection of materials in \cref{part:gka}, the anharmonicity score by Knoop~\etal~\cite{kpsc2020t} has been employed. While the value for \ch{SnSe} at \qty{300}{K} has been obtained from reference~\cite{k2021t}, values for \ch{Si} at \qty{400}{K} and \ch{ZrO2}at \qty{300}{K} and \qty{1400}{K} were computed.

For \ch{ZrO2}, anharmonicity scores were obtained based on the harmonic sampling technique from reference~\cite{kpsc2020t}, using \aims and \num{30} samples in the \num{324}-atom supercell used for phonon band structure calculations in \cref{fig:gkm_phonons}. 

For \ch{Si}, $\anha$ was computed based on an \aimd trajectory of \qty{4}{ps} duration, using the same settings as the training data used for \ch{Si} in \cref{ch:gk-out}. Force constants were obtained from the phonon band structure calculation reported in \cref{fig:gko_si_phonons}.

\clearpage
\section{Model for Zirconia}

\subsection{Training Data}
\begin{figure}
  \includegraphics[width=0.92\textwidth]{img/plot/gk/si-zro_training_data_cells.pdf}
  \caption{
  Magnitude of lattice vectors for the zirconia $NpT$ dataset. The vertical line separates the samples used for training from those used for validation.
	}
	\label{fig:si-gkm_training_cells}
\end{figure}

\clearpage
\subsection{Vibrational Density of States}
\begin{figure}
  \centering
  \caption[][-1\baselineskip]{
  Comparison of \gls{vdos} for \mpnns{} ($\interactions{=}1$) with different cutoff radii compared to a baseline computed with \aims. The chosen production cutoff radius is highlighted in red.
  Vertical lines indicate peaks in the \aims result. Constant vertical offsets are applied to distinguish curves. Results are averaged over three trajectories of \si{60}{ps} ($\Delta t{=}\SI{4}{fs}$), in the tetragonal phase at \SI{300}{K}, with matching initial configurations. Shaded areas indicate the minimum and maximum.
  \\\\
  The corresponding figure with $\interactions{=}2$ can be found in \cref{fig:gkm_vdos}.
  }
  \label{fig:si-gkm_vdos_t1}
  \includegraphics[width=\textwidth]{img/plot/gk/zro_vdos_t1.pdf}
\end{figure}

\clearpage
\subsection{Phonon Band Structures}

\begin{figure}
  \centering
  \subfloat[Monoclinic]{
    \includegraphics[width=\textwidth]{img/plot/gk/si-zro_phonons_mono_m1.pdf}
    \label{}
  }

  \subfloat[Tetragonal]{
    \includegraphics[width=\textwidth]{img/plot/gk/si-zro_phonons_tetra_m1.pdf}
    \label{}
  }

  \caption{
  Phonon band structure and density of states in the monoclinic (top) and tetragonal (bottom) phases, using alternate \schnet \mlps with $\interactions{=}1$ and different cutoff radii, compared to a \aims reference calculation.
  Results are shown for a 324-atom supercell. Convergence with respect to supercell size was checked.
  \\\\
  Production settings, $\interactions{=}2$ and $\cutoff{=}\qty{5.0}{\angstrom}$ are shown in \cref{fig:gkm_phonons}.  
  }
  \label{fig:si-gkm_phonons_m1}
\end{figure}

\clearpage
\begin{figure}
  \centering
  \subfloat[Monoclinic]{
    \includegraphics[width=\textwidth]{img/plot/gk/si-zro_phonons_mono_m2.pdf}
    \label{}
  }

  \subfloat[Tetragonal]{
    \includegraphics[width=\textwidth]{img/plot/gk/si-zro_phonons_tetra_m2.pdf}
    \label{}
  }

  \caption{
  Phonon band structure and density of states in the monoclinic (top) and tetragonal (bottom) phases, using alternate \schnet \mlps with $\interactions{=}2$ and different cutoff radii, compared to a \aims reference calculation.
  Results are shown for a 324-atom supercell. Convergence with respect to supercell size was checked.
  \\\\
  Production settings, $\interactions{=}2$ and $\cutoff{=}\qty{5.0}{\angstrom}$ are shown in \cref{fig:gkm_phonons}.  
  }
  \label{fig:si-gkm_phonons_m2}
\end{figure}

\vspace*{\fill}

\clearpage
\subsection{Displacements}


\begin{figure}
  \includegraphics[width=\textwidth]{img/plot/gk/zro_displacements_schnet_mono.pdf}
  \caption{
  Maximum displacement across oxygen atoms with respect to average positions over the course of a single trajectory at different temperatures in the monoclinic phase, computed with the $\interactions{=}2$ and $\cutoff{=}\qty{5.0}{\angstrom}$ production SchNet \mlp.
  The top plot shows the full range of data, while the lower displays details.
  }
  \label{fig:si-gkm_disp_schnet_mono}
\end{figure}


\begin{figure}
  \includegraphics[width=\textwidth]{img/plot/gk/si-zro_displacements_schnet_tetra_hi.pdf}
  \caption{
  Maximum displacement across oxygen atoms with respect to average positions over the course of a single trajectory at different temperatures in the tetragonal phase, computed with the $\interactions{=}2$ and $\cutoff{=}\qty{5.0}{\angstrom}$ production SchNet \mlp.
  
  This figure includes a trajectory at \qty{2000}{K} where an instability is encountered: After significant oxygen diffusion, the potential becomes unstable and displacements, along with energy and temperature, diverge.
  }
  \label{fig:si-gkm_disp_schnet_tetra_hi}
\end{figure}


\begin{figure}
  \includegraphics[width=\textwidth]{img/plot/gk/zro_displacements_schnet_tetra.pdf}
  \caption{
  Maximum displacement across oxygen atoms with respect to average positions over the course of a single trajectory at different temperatures in the tetragonal phase, computed with the $\interactions{=}2$ and $\cutoff{=}\qty{5.0}{\angstrom}$ production SchNet \mlp.
  The top plot shows the full range of data, while the lower displays details.
  }
  \label{fig:si-gkm_disp_schnet_tetra}
\end{figure}


\begin{figure}
  \includegraphics[width=\textwidth]{img/plot/gk/zro_displacements_aims.pdf}
  \caption{
  Maximum displacement across oxygen atoms with respect to average positions for the trajectories in the training data, computed with \aims.
  The top plot shows the full range of data, while the lower displays details.
  }
  \label{fig:si-gkm_disp_aims}
\end{figure}


\clearpage
\section{Green-Kubo Convergence}

% \begin{figure}
%   \includegraphics[width=\textwidth]{img/plot/gk/zro_kc_ens_t_1400.pdf}
%   \caption{
%   $\kt$ for tetragonal zirconia at \qty{1400}{K}, comparing ensemble average and single trajectories.
%   Opaque lines indicate $\kt$ with applied noise reduction, translucent ones without.
%   \\\\
%   Figure for \qty{300}{K} in \cref{fig:gkc_ens_m_300}.
%   }
%   \label{fig:si-gkc_ens_t_1400}
% \end{figure}
% \clearpage


\subsection{Noise Removal}

\begin{figure*}
  \includegraphics[width=\textwidth]{img/plot/gk/zro_kc_noise_parts_m_300_unconverged.pdf}
  \caption{
  $\kt$ (top) and $\ehfacf(\tau)$ (bottom) for monoclinic zirconia at \qty{300}{K} for \scare{unconverged} parameter choices $\coco{96}{0.1}$, comparing different components of the noise reduction approach.
  The vertical lines in the bottom plot indicate the cutoff time.
  \\\\
  Figure for \scare{production} settings in \cref{fig:gkc_hfacf_noise_comparison}.
  }
  \label{fig:si-gkc_hfacf_noise_comparison_unconverged}
\end{figure*}

\clearpage

\begin{figure*}
  \includegraphics[width=\textwidth]{img/plot/gk/zro_kc_noise_parts_m_1400_converged.pdf}
  \caption{
  $\kt$ (top) and $\ehfacf(\tau)$ (bottom) for monoclinic zirconia at \qty{1400}{K} for \scare{production} parameter choices $\zrocc$, comparing different components of the noise reduction approach.
  The vertical lines in the bottom plot indicate the cutoff time.
  }
  \label{fig:si-gkc_hfacf_noise_comparison_converged_m_1400}
\end{figure*}

\clearpage
\subsection{Filter Frequency}

\begin{figure}
  \includegraphics[width=\textwidth]{img/plot/gk/zro_kc_freq_m_300_converged.pdf}
  \caption{
  $\kt$ and $\kappa$ as determined from the \hfacf for different choies of $\omegaf$ 
  for monoclinic zirconia at \qty{300}{K}
  with \scare{production} settings $\zrocc$.
  Shaded areas indicate standard error over trajectories, lines the mean.
  }
  \label{fig:si-gkc_freq_m_300_converged}
\end{figure}

\begin{figure}
  \includegraphics[width=\textwidth]{img/plot/gk/zro_kc_freq_t_1400_converged.pdf}
  \caption{
  $\kt$ and $\kappa$ as determined from the \hfacf for different choies of $\omegaf$ 
  for tetragonal zirconia at \qty{1400}{K}
  with \scare{production} settings $\zrocc$.
  Shaded areas indicate standard error over trajectories, lines the mean.
  }
  \label{fig:si-gkc_freq_t_1400_converged}
\end{figure}

% more logical grouping imho
\clearpage
\begin{figure}
  \includegraphics[width=\textwidth]{img/plot/gk/zro_kc_freq_m_300_semi.pdf}
  \caption{
  $\kt$ and $\kappa$ as determined from the \hfacf for different choies of $\omegaf$ 
  for monoclinic zirconia at \qty{300}{K}
  with \scare{light} settings $\zrosc$.
  Shaded areas indicate standard error over trajectories, lines the mean.
  }
  \label{fig:si-gkc_freq_m_300_semi}
\end{figure}

\begin{figure}
  \includegraphics[width=\textwidth]{img/plot/gk/zro_kc_freq_t_1400_semi.pdf}
  \caption{
  $\kt$ and $\kappa$ as determined from the \hfacf for different choies of $\omegaf$ 
  for tetragonal zirconia at \qty{1400}{K}
  with \scare{light} settings $\zrosc$.
  Shaded areas indicate standard error over trajectories, lines the mean.
  }
  \label{fig:si-gkc_freq_t_1400_semi}
\end{figure}

\clearpage
\subsection{Spacing}

\begin{figure}
  \includegraphics[width=\textwidth]{img/plot/gk/zro_kc_spacing_m_300_semi.pdf}
  \caption{
  $\kt$ and final $\kappa$ for different choies of $\nhf$ 
  for monoclinic zirconia at \qty{300}{K}
  with \scare{light} settings $\zrosc$.
  The upper edges of the labels indicating $\nhf$ are aligned with horizontal lines indicating the value for $\kappa$ chosen by the first dip of the corresponding \hfacf.
  }
  \label{fig:si-gkc_spacing_m_300_semi}
\end{figure}

\begin{figure}
  \includegraphics[width=\textwidth]{img/plot/gk/zro_kc_spacing_m_1400_semi.pdf}
  \caption{
  $\kt$ and final $\kappa$ for different choies of $\nhf$ 
  for monoclinic zirconia at \qty{1400}{K}
  with \scare{light} settings $\zrosc$.
  The upper edges of the labels indicating $\nhf$ are aligned with horizontal lines indicating the value for $\kappa$ chosen by the first dip of the corresponding \hfacf.
  }
  \label{fig:si-gkc_spacing_m_1400_semi}
\end{figure}

\clearpage
\subsection{Number of Trajectories}

\begin{figure}
  \includegraphics[width=\textwidth]{img/plot/gk/zro_kc_ens_conv_m_300_converged.pdf}
  \caption{
  $\kt$ and final $\kappa$ for different number of trajectories $n$
  for monoclinic zirconia at \qty{300}{K}
  with \scare{production} settings $\zrocc$.
  Horizontal lines indicate the value for $\kappa$ chosen by the first dip of the corresponding \hfacf.
  }
  \label{fig:si-gkc_ens_conv_m_300_converged}
\end{figure}

\begin{figure}
  \includegraphics[width=\textwidth]{img/plot/gk/zro_kc_ens_conv_t_1400_converged.pdf}
  \caption{
  $\kt$ and final $\kappa$ for different number of trajectories $n$
  for tetragonal zirconia at \qty{1400}{K}
  with \scare{production} settings $\zrocc$.
  Horizontal lines indicate the value for $\kappa$ chosen by the first dip of the corresponding \hfacf.
  }
  \label{fig:si-gkc_ens_conv_t_1400_converged}
\end{figure}


\clearpage
\subsection{Size and Time}

\begin{figure*}
  \includegraphics[width=\textwidth]{img/plot/gk/zro_kc_nt_m_1400.pdf}
  \caption{
  $\kappa$ for monoclinic zirconia at \qty{1400}{K} for different choices of $N$ and $\td$.
  Error bars indicate the standard error across trajectories.
  $N$ is shown as $N^{1/3}$, which is proportional to the length scale of the simulation cell.
  For each choice of $N$, $\td$ from \qtyrange{0.1}{2.0}{ns} are shown with a horizontal offset.
  \scare{Production}, \scare{light}, \scare{unconverged} and \scare{tight} choices are indicated; for the \scare{production} setting, the associated standard error is also shown as a shaded band.
  \\\\
  \Cref{fig:gkc_nt} shows \qty{300}{K}.
  }
  \label{fig:si-gkc_nt_m_1400}
\end{figure*}


\clearpage
\begin{figure*}
  \includegraphics[width=\textwidth]{img/plot/gk/zro_kc_nt_t_1400.pdf}
  \caption{
  $\kappa$ for tetragonal zirconia at \qty{1400}{K} for different choices of $N$ and $\td$.
  Error bars indicate the standard error across trajectories.
  $N$ is shown as $N^{1/3}$, which is proportional to the length scale of the simulation cell.
  For each choice of $N$, $\td$ from \qtyrange{0.1}{2.0}{ns} are shown with a horizontal offset.
  \scare{Production}, \scare{light}, \scare{unconverged} and \scare{tight} choices are indicated; for the \scare{production} setting, the associated standard error is also shown as a shaded band.
  }
  \label{fig:si-gkc_nt_t_1400}
\end{figure*}

\noindent
The outliers at low $\td$ for $N{=}4116$ in \cref{fig:si-gkc_nt_t_1400} are due to the \hfacf narrowly avoiding a zero crossing that emerges at higher $\td$.
It highlights that results for singular choices of $N$ and $\td$ must be carefully investigated, as they may emerge from such an artefact.
The impact of such events can be reduced by choosing $\tc$ not as the first zero crossing, $\ehfacf(\tc) = 0$, but rather at a small positive number, $\keps$, such that $\ehfacf(\tc){=}\keps$. In this thesis, this is not done to avoid an additional parameter in the \gk method.

Alternatively, more sophicistaced methods for the determination of $\tc$ can be employed, for instance by fitting a functional form~\cite{mk2004t,ccdg2000t}.
$\kappa$ can also be determined by averaging over $\kt$ after $\tc$~\cite{fpdh2015t}, or via spectral methods~\cite{emb2017t}.


\clearpage
\section{Gauge Term}
\label{sec:si-gkc_gauge_term}

As shown by Knoop~\etal~\cite{ksc2023t}, it is advantageous to subtract a non-diffusive \scare{gauge term} from the instantaneous heat flux $\J(t)$ to reduce noise.
This term takes the form
\begin{equation}
	\J^{\text{gauge}} = \sum_{i \in \Rsc} \virial_i \cdot \V_i
\end{equation}
with a constant $3 \times 3$ matrix $\virial_i$ for each atom, which is computed from the terms preceding the velocities in $\Jpot$ in \cref{eq:hf_general}. 
As $\virial_i$ is constant, $\J^{\text{gauge}}$ is a total time derivative, and by the gauge principle, does not contribute to the final thermal conductivity.
However, as seen in \cref{sec:gkc_noise}, removing this term from $\J$ can reduce noise.

In the original formulation, the \scare{virials}\footnote{We note in passing that in this definition, the terms $\virial_i$ do not add up to the total stress in all cases.
% TODO: add reference here to where we discuss this
} are computed as time averages over the trajectory, 
\begin{equation}
	\virial_i^{\text{original}} = \braket{\R_{ji} \otimes \dur{i}{j}{}} \, .
\end{equation}
As has been discussed in \cref{ch:gk-hf}, an efficient heat flux formulation for \glps does not necessarily yield $\indur{i}{j}{}$ at every timestep, as the heat flux is directly computed with \ad.
% It is therefore advantageous to avoid computing $\virial_i$ at each timestep. 
Instead, we compute $\virial_i$ once for a reference structure, and in particular the supercell for a given temperature and phase as obtained in \cref{sec:si-gknet_zro_lattice}. In this formulation, $\Jpot$ and $\Jint$ can be treated interchangeably, assuming that the reference structure used for $\Rr$ is the one used for $\virial_i$.

These two approaches are compared in \cref{fig:si-gkc_virial_m_300,fig:si-gkc_virial_m_1400}. No significant differences between both methods are observed. Indeed, if filtering is applied (not shown), any remaining differences vanish entirely.

\clearpage
\begin{figure}
  \includegraphics[width=\textwidth]{img/plot/gk/si-zro_gkc_virial_m_300.pdf}
  \caption{
  $\kt$ for differently computed gauge terms
  for monoclinic zirconia at \qty{300}{K}
  with \scare{light} settings $\zrosc$.
  No filtering is applied. Heat flux is computed every $\nhf{=}2$ simulation steps.
  }
  \label{fig:si-gkc_virial_m_300}
\end{figure}

\begin{figure}
  \includegraphics[width=\textwidth]{img/plot/gk/si-zro_gkc_virial_m_1400.pdf}
  \caption{
  $\kt$ for differently computed gauge terms
  for monoclinic zirconia at \qty{1400}{K}
  with \scare{light} settings $\zrosc$.
  No filtering is applied. Heat flux is computed every $\nhf{=}2$ simulation steps.
  }
  \label{fig:si-gkc_virial_m_1400}
\end{figure}


\clearpage
\clearpage
\section{Potential and Convective Flux}
\label{sec:si-gkc_conv}

In \cref{sec:hf_terms}, we discussed the distinction between the decomposition of the total heat flux into
\begin{equation}
	\J = \Jpot + \Jconv
\end{equation}
and
\begin{equation}
	\J = \Jint + \Jdiff \, ,
\end{equation}
finding that $\J$ and $\Jint$ should be equivalent if atomic positions are bounded over the course of the simulation.
This is verified for zirconia in \cref{fig:si-gkc_convective_converged}; $\Jint$ and $\J$ are found to be equivalent, while $\Jpot$ is not. $\Jconv$ is non-vanishing, but yields a significantly smaller thermal conductivity on its own.
At increased temperature, where atoms become more mobile, the magnitude of $\Jconv$ and hence the difference between $\J$ and $\Jpot$ becomes more pronounced.

We can therefore conclude that $\Jint$ can be used in place of $\J$ for zirconia at the investigated range of temperatures and phases.

\clearpage
\begin{figure}
  \centering
  \subfloat[Monoclinic, \qty{300}{K}]{
    \includegraphics[width=\textwidth]{img/plot/gk/zro_kc_convective_m_300_converged.pdf}
    \label{}
  }

  \subfloat[Tetragonal, \qty{1400}{K}]{
    \includegraphics[width=\textwidth]{img/plot/gk/zro_kc_convective_t_1400_converged.pdf}
    \label{}
  }

  \subfloat[Tetragonal, \qty{1800}{K}]{
    \includegraphics[width=\textwidth]{img/plot/gk/zro_kc_convective_t_1800_converged.pdf}
    \label{}
  }

  \caption{
  $\kt$ for different terms in the heat flux
  for zirconia at different temperatures and in different phases,
  using \scare{production} settings $\zrocc$.
  Filtering with $\omegaf{=}\qty{1}{THz}$ is applied; no gauge term is removed.
  }
  \label{fig:si-gkc_convective_converged}
\end{figure}


\clearpage
\section{Lattice Vectors and Supercells}
\label{sec:si-gknet_zro_lattice}

As discussed in \cref{ch:gk-model}, the employed \mlp displays limited accuracy when predicting lattice constants and treating the monoclinic to tetragonal phase transition.
Since the focus of the present work is the heat flux, and since results are compared to experimental measurements, experimental lattice parameters are therefore used.
The used lattice parameters, as presented by Verdi~\etal{}~\cite{vkjk2021q}, based on references~\cite{ps1969t,kh1998t}, can be found in \cref{tab:si-zro_cells_mono,tab:si-zro_cells_tetra}.

The workflow to obtain supercells at a given temperature $T$ and number of atoms $N$ is as follows: The starting point is always a 12-atom unit cell. Lattice parameters are applied to this original cell, leaving atomic positions in fractional coordinates unchanged. From the resulting cell, simple $k \times k \times k$ supercells are constructed such that $N = 12 k^3$.

\vspace{2\baselineskip}

\begin{table}
    \caption{Lattice parameters for monoclinic zirconia based on references~\cite{ps1969t,kh1998t,vkjk2021q}.}
	\begin{tabular}{l|rrrr}
\toprule
    Temperature in K  &  $a$ in \unit{\angstrom}  &  $b$ in \unit{\angstrom}  &  $c$ in \unit{\angstrom}  &  $\beta$ in \unit{\degree} \\ 
\midrule
           \num{300}  &           \num{5.147}  &           \num{5.209}  &           \num{5.311}  &           \num{99.30} \\ 
           \num{350}  &           \num{5.150}  &           \num{5.209}  &           \num{5.315}  &           \num{99.28} \\ 
           \num{400}  &           \num{5.152}  &           \num{5.211}  &           \num{5.318}  &           \num{99.27} \\ 
           \num{450}  &           \num{5.154}  &           \num{5.211}  &           \num{5.322}  &           \num{99.26} \\ 
           \num{600}  &           \num{5.161}  &           \num{5.213}  &           \num{5.330}  &           \num{99.17} \\ 
           \num{750}  &           \num{5.165}  &           \num{5.213}  &           \num{5.340}  &           \num{99.07} \\ 
           \num{900}  &           \num{5.172}  &           \num{5.215}  &           \num{5.349}  &           \num{98.99} \\ 
          \num{1050}  &           \num{5.179}  &           \num{5.216}  &           \num{5.356}  &           \num{98.86} \\ 
          \num{1200}  &           \num{5.193}  &           \num{5.237}  &           \num{5.386}  &           \num{98.74} \\ 
          \num{1400}  &           \num{5.196}  &           \num{5.219}  &           \num{5.388}  &           \num{98.69} \\ 
\bottomrule
\end{tabular}

    \label{tab:si-zro_cells_mono}
\end{table}

\vspace{2\baselineskip}

\begin{table}
    \caption{Lattice parameters for tetragonal zirconia based on references~\cite{ps1969t,kh1998t,vkjk2021q}.}
	\begin{tabular}{l|rr}
\toprule
    Temperature in K  &  $a$ in \unit{\angstrom}  &  $b$ in \unit{\angstrom} \\ 
\midrule
          \num{1400}  &           \num{5.149}  &           \num{5.278} \\ 
          \num{1500}  &           \num{5.155}  &           \num{5.284} \\ 
          \num{1650}  &           \num{5.165}  &           \num{5.295} \\ 
          \num{1800}  &           \num{5.174}  &           \num{5.305} \\ 
\bottomrule
\end{tabular}

    \label{tab:si-zro_cells_tetra}
\end{table}

\clearpage

\section{Additional Results for Zirconia}


\begin{figure}
  \includegraphics[width=\textwidth]{img/plot/gk/zro_hfs_all_m_300.pdf}
  \caption{
  $\kt$ for all heat flux formulations for monoclinic zirconia at \qty{300}{K},
  using a SchNet \glp ($\interactions{=\,}2$, $\cutoff{=}\qty{5}{\angstrom}$).
  See \cref{fig:gkr_hfs_m_300} for details.
  }
  \label{fig:si-gkr_hfs_all_m_300}
\end{figure}
\vspace{4\baselineskip}

\begin{figure}
  \includegraphics[width=\textwidth]{img/plot/gk/zro_hfs_all_m1_m_300.pdf}
  \caption{
  $\kt$ for all heat flux formulations for monoclinic zirconia at \qty{300}{K},
  using a SchNet \glp ($\interactions{=\,}1$, $\cutoff{=}\qty{5}{\angstrom}$).
  See \cref{fig:gkr_hfs_m1_m_300} for details.
  % \\\\
  % The corresponding result for the monoclinic phase can be found in \cref{fig:gkr_hfs_all}.
  }
  \label{fig:si-gkr_hfs_all_m1_m_300}
\end{figure}

\clearpage
\section{Measurements of Thermal Conductivities of Zirconia}
\label{sec:si-gknet_zro_exp}

\noindent
A number of experimental measurements of the thermal conductivity of \zro{} have been undertaken.~\cite{hjls1987t,rwpm1998t,bflm2000t,mlld2004t} 
In all cases, $\kappa$ is determined indirectly, based on measurements of the thermal diffusivity $\alpha$, the specific heat capacity $c$, and the density $\rho$, as $\kappa = \alpha \rho c$.
\marginnote[\baselineskip]{The standard method for measuring thermal diffusivity, used in references~\cite{hjls1987t,rwpm1998t,mlld2004t}, is the \scare{laser flash technique}, where one end of a sample is briefly heated, and the temperature change at the opposite end is observed. However, the method probes a large area of the sample, which may include imperfections that can hinder thermal transport, prompting the development of alternative techniques, such as the \scare{spatially resolved infrared thermography} method used in~\cite{bflm2000t}.}

Hasselman~\etal{}~\cite{hjls1987t} investigate polycrystalline samples with varying \ch{MgO} (magnesia) and \ch{Y2O3} (yttria) content at room temperature, estimating results for pure samples in different phases with an ansatz based on composite theory. For the tetragonal phase, $\kappa = \SI{4.82}{W\per m K}$, and for the monoclinic phase $\kappa = \SI{5.2}{W\per m K}$, indicating that different phases, at least at room temperature, display comparable thermal conductivities. As this study does not use pure samples, it has not been included in \cref{fig:gkr_kappa_vs_temp}.

Raghavan~\etal{}~\cite{rwpm1998t} measure pure and yttria-stabilised \zro{} nanoparticles in the monoclinic phase from \SI{100}{\celsius} to \SI{1000}{\celsius}. They observe a strong, approximately $\nicefrac{1}{T}$, temperature dependence of $\kappa$ for the pure sample, but not for samples with significant yttria content.

Bisson~\etal{}~\cite{bflm2000t} study pure and yttria-stabilised single crystals at room temperature, measuring $\kappa = \SI{8.2}{W\per m K}$ for pure monoclinic zirconia.\footnote[][-1\baselineskip]{An uncertainty of \scare{the order of \SI{10}{\percent}} is mentioned.} The higher value for $\kappa$ obtained here may be attributed to a different methodology for determining $\alpha$, which is less sensitive to grain boundaries. No significant anisotropy is found for $\kappa$.

Finally,\footnote[][-2\baselineskip]{Youngblood~\etal{}~\cite{yri1988t} is not included in the summary, as the work does not consider undoped zirconia. We also exclude studies of porous samples.} Mévrel~\etal{}~\cite{mlld2004t} determine $\kappa$ for pure (monoclinic) and yttria-stabilised (tetragonal, cubic) single crystals from room temperature to \SI{1100}{\celsius}. An amended model\footnote{The model, based on the assumption that the phonon mean free path is bounded from below by the minimum neighbour distance, gives $\kappa(T) = \nicefrac{A}{T} \left[\nicefrac{2 \sqrt{T_1}}{{3\sqrt{T}}} + \nicefrac{T}{3 T_1} \right]$ with fit parameters $A$ and $T_1$.}
 for the temperature dependence of $\kappa$ is also given.

\begin{margintable}
    \centering
    \begin{tabular}{r r r r r}
    \toprule
     & Phase & Single & Pure & $\kappa(T)$\\
    \midrule
    \cite{hjls1987t}  & t,m & \no  & \no  &  \no \\
    \cite{rwpm1998t}   & m   & \no  & \yes &  \yes\\
    \cite{bflm2000t}   & m   & \yes & \yes &  \no \\
    \cite{mlld2004t} & m   & \yes & \yes &  \yes\\
    \bottomrule
    \end{tabular}
    
    \caption{Experiments measuring $\kappa$ for pure \zro{}. \emph{Single} = single crystal samples used, \emph{pure} = undoped sample directly measured, $\kappa(T)$ = measured temperature dependence.}
    \label{tab:zirconia-experiments}
\end{margintable}


\newthought{Measurement results for} the thermal conductivity of \zro{} are included in \cref{fig:gkr_kappa_vs_temp}, experimental setups are tabulated in \cref{tab:zirconia-experiments}. At low temperatures, references~\cite{rwpm1998t,bflm2000t,mlld2004t} report similar values. Hasselman~\etal{} provide a lower estimate, which is, however, not based on direct measurement, and therefore not included. As temperature increases, Mévrel~\etal{} measure a more rapid decrease in $\kappa$ compared to Raghavan~\etal{}, despite performing measurements on single crystals. The reason for this behaviour is unclear, as polycrystalline samples are expected to display reduced thermal conductivity due to phonon scattering at grain boundaries.
 % The overall temperature-dependence of $\kappa$ is consistent with dominant phonon-phonon scattering.

\clearpage

\section{Tin Selenide}
\label{sec:si-snse}

\subsection{Testing}

\begin{table}
\label{tab:si-snse_errors_m1}
\caption{Test set errors for \sok for \ch{SnSe}, with $\interactions{=}1$.}
\begin{tabular}{l | r r r r}
\toprule
Property & \acs{rmse} & \acs{mae} & \acs{maxae} & $R^2$ in \unit{\percent} \\
\midrule
Energy in \unit{meV}  & \num{94.5029} & \num{73.6968} & \num{307.2220}  & \num{95.8476} \\
Forces in \unit{meV/\angstrom} & \num{52.1557} & \num{41.3036} & \num{321.0409} & \num{97.4306} \\
\bottomrule
% Stress in \unit{kbar}  & 13166.1118 / MAE: 8592.0826 / maxAE: 60126.4910 / R$^2$: -2.3029
\end{tabular}
\end{table}

\begin{table}
\label{tab:si-snse_errors_m2}
\caption{Test set errors for \sok for \ch{SnSe}, with $\interactions{=}2$.}
\begin{tabular}{l | r r r r}
\toprule
Property & \acs{rmse} & \acs{mae} & \acs{maxae} & $R^2$ in \unit{\percent} \\
\midrule
Energy in \unit{meV} & \num{28.2936} & \num{22.4426} & \num{78.1010} & \num{99.6278} \\
Forces in \unit{meV/\angstrom} & \num{19.7130} & \num{15.5923} & \num{148.4874} & \num{99.6329} \\
\bottomrule
% Stress in \unit{kbar} & 13167.5711 / MAE: 8592.9597 / maxAE: 60128.7621 / R$^2$: -2.3255
\end{tabular}
\end{table}

\begin{table}
\label{tab:si-snse_errors_m3}
\caption{Test set errors for \sok for \ch{SnSe}, with $\interactions{=}3$.}
\begin{tabular}{l | r r r r}
\toprule
Property & \acs{rmse} & \acs{mae} & \acs{maxae} & $R^2$ in \unit{\percent} \\
\midrule
Energy in \unit{meV} & \num{17.0437} & \num{13.4452} & \num{47.9379} & \num{99.8649} \\
Forces in \unit{meV/\angstrom} & \num{13.5658} & \num{10.7136} & \num{85.0041 } & \num{99.8262} \\
\bottomrule
% Stress in \unit{kbar} & 13166.8269 / MAE: 8592.5370 / maxAE: 60130.7086 / R$^2$: -2.3140
\end{tabular}
\end{table}

\vspace{\baselineskip}

\begin{figure}
  \includegraphics[width=\textwidth]{img/plot/gk/snse_vdos.pdf}
  \caption{
  Comparison of \gls{vdos} in \ch{SnSe} for \sok with varying values of $\interactions$ compared to a baseline computed with \aims.
  Vertical lines indicate peaks in the \aims result.
  Results are shown for one trajectory of \si{30}{ps} ($\Delta t{=}\SI{4}{fs}$) at \SI{300}{K} in a supercell containing \num{256} atoms, with matching initial configurations.
  }
  \label{fig:gko_snse_vdos}
\end{figure}


\begin{figure}
  \includegraphics[width=\textwidth]{img/plot/gk/snse_phonons.pdf}
  \caption{
  Comparison of phonon band structure and density of states in \ch{SnSe} for \sok with varying values of $\interactions$ compared to a baseline computed with \aims.
  Results are showsn for a supercell containing \num{256} atoms, with matching lattice and positions relaxed to the \qty{0}{K} configuration.
  }
  \label{fig:gko_snse_phonons}
\end{figure}

\clearpage
\subsection{Green-Kubo Convergence}

\begin{figure}
  \includegraphics[width=\textwidth]{img/plot/gk/snse_kappa_convergence_N_t.pdf}
  \caption{
  $\kappa$ for tin selenide at \qty{300}{K} for different choices of $N$ and $\td$ using a \sok model with $\interactions{=}2$.
  Error bars indicate the standard error across trajectories.
  $N$ is shown as $N^{1/3}$, which is proportional to the length scale of the simulation cell.
  For each choice of $N$, $\td$ from \qtyrange{0.1}{4.0}{ns} are shown with a horizontal offset.
  \scare{Production} settings are indicated, the associated standard error is also shown as a shaded band.
  }
  \label{fig:gko_snse_kappa_convergence_N_t}
\end{figure}

\clearpage
\section{Silicon}
\label{sec:si-si}

\begin{table}
\label{tab:si-si_errors_m1}
\caption{Test set errors for \sok for \ch{Si}, with $\interactions{=}1$.}
\begin{tabular}{l | r r r r}
\toprule
Property & \acs{rmse} & \acs{mae} & \acs{maxae} & $R^2$ in \unit{\percent} \\
\midrule
Energy in \unit{meV} & \num{92.4688} & \num{71.8210} & \num{295.3312} & \num{99.9784} \\
Forces in \unit{meV/\angstrom} & \num{45.5592} & \num{35.1807} & \num{330.7921} & \num{99.7029} \\
Stress in \unit{kbar} & \num{0.8973} & \num{0.6349} & \num{2.5944} & \num{99.8320} \\
\bottomrule
\end{tabular}
\end{table}

\begin{table}
\label{tab:si-si_errors_m2}
\caption{Test set errors for \sok for \ch{Si}, with $\interactions{=}2$.}
\begin{tabular}{l | r r r r}
\toprule
Property & \acs{rmse} & \acs{mae} & \acs{maxae} & $R^2$ in \unit{\percent} \\
\midrule
Energy in \unit{meV} & \num{42.3110} & \num{33.1063} & \num{153.6074} & \num{99.9955} \\
Forces in \unit{meV/\angstrom} & \num{23.6281} & \num{18.0433} & \num{274.2255} & \num{99.9201} \\
Stress in \unit{kbar} & \num{0.9403} & \num{0.6905} & \num{2.2113} & \num{99.8155} \\
\bottomrule
\end{tabular}
\end{table}

\begin{table}
\label{tab:si-si_errors_m3}
\caption{Test set errors for \sok for \ch{Si}, with $\interactions{=}3$.}
\begin{tabular}{l | r r r r}
\toprule
Property & \acs{rmse} & \acs{mae} & \acs{maxae} & $R^2$ in \unit{\percent} \\
\midrule
Energy in \unit{meV} & \num{34.6730} & \num{25.9488} & \num{150.2103} & \num{99.9970} \\
Forces in \unit{meV/\angstrom} & \num{16.7011} & \num{12.6485} & \num{142.6790} & \num{99.9601} \\
Stress in \unit{kbar} & \num{0.9720} & \num{0.6784} & \num{2.9101} & \num{99.8029} \\
\bottomrule
\end{tabular}
\end{table}

\vspace{2\baselineskip}

\begin{figure}
  \includegraphics[width=\textwidth]{img/plot/gk/si_phonons.pdf}
  \caption{
  Comparison of phonon band structure and density of states in \ch{Si} for \sok with varying values of $\interactions$ compared to a baseline computed with \aims.
  Results are showsn for a supercell containing \num{216} atoms, with matching lattice and positions relaxed to the \qty{0}{K} configuration.
  }
  \label{fig:gko_si_phonons}
\end{figure}
