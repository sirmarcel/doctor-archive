%!TEX root = ../../da.tex

\clearpage


\part{Foundations}

% You gotta build a FOUNDATION
\label{part:foundation}

\thispagestyle{plain}
\begin{center}
  \begin{minipage}{0.8\textwidth}

    \vspace{4\baselineskip}

    You got to build your \textit{\textbf{foundations}!}

    \vspace{\baselineskip}
    {\hfill\raggedright -- overheard in Manchester pub\hspace{0.25cm}}
  \end{minipage}
\end{center}
\vspace{2\baselineskip}

\newthought{This chapter sets the stage} for the work presented in \cref{part:representations,part:hf,part:gka} by providing an overview of foundational concepts and introducing relevant terminology and notation.

In \cref{ch:qm}, we begin by discussing quantum mechanics, and the particular task of solving the electronic Schrödinger equation. Computational approaches to this task are briefly discussed, with a particular focus on \gls{dft}. We conclude with the introduction of the \bo \pes.
The dynamics of nuclei moving on this \pes can be modelled with \md, the focus of \cref{ch:md}. Approaches to modeling systems at different thermodynamic conditions are introduced, and the usage of approximate \pes such as \ffs and \mlps is motivated.
\Cref{ch:pbc} then explores how bulk properties can be obtained from simulations with a finite number of independent atoms through periodicity.
% As the particular treatment of periodicity is of great importance for \cref{part:gk}, a particular focus is put on practical issues of describing periodic systems.
\Cref{ch:ml} introduces the concepts and terminology of \ml applied to atomistic modeling, as well as providing an overview of the techniques and types of applications relevant for this thesis.
Finally, \cref{ch:gk} introduces the \gk method for the calculation of thermal conductivity.
