%!TEX root = ../../da.tex

%  via Automatic Differentiation
\part{Heat Flux for Semi-Local Machine-Learning Potentials}
\label{part:hf}

\thispagestyle{plain}
\begin{center}
  \begin{minipage}{0.8\textwidth}

    \vspace{4\baselineskip}
    
    \hangindent=1cm
    \hangafter=1
    I saw the best minds of my generation destroyed by madness, starving hysterical naked, \el

    \hangindent=1cm
    \hangafter=1
    who passed through universities with radiant cool eyes hallucinating Arkansas and Blake-light tragedy among the scholars of war, \el


    \vspace{\baselineskip}
    {\hfill\raggedright --Allen Ginsberg, \textit{Howl}\hspace{0.25cm}}
  \end{minipage}
\end{center}
\vspace{2\baselineskip}

\newthought{The \gk method~\cite{g1952t,k1957t,kyn1957t}}, briefly introduced in \cref{ch:gk}, is a framework for the prediction of the thermal conductivity of materials, including strongly anharmonic and complex compounds.
It is based on observing fluctuations in the distribution of energy, the \newterm{heat flux}, during equilibrium \md simulations.
It therefore requires access to the \bo \pes as well as a definition of the heat flux.

\ffs have been used for the \gk method since the 1970s~\cite{lvk1973t,lmh1986t,vc1999t,c2006t,ggs2010t,fpdh2015t}, decades prior to the recent development of \aigk approaches~\cite{mub2016t,crs2017t}.
\mlps continue this tradition, aiming to combine computational efficiency, albeit typically reduced in comparison to \ffs, with the accuracy of \dft.
While different types of short-ranged \mlps{} have been used to investigate thermal transport via \gk{}~\cite{sdbb2012q,mcld2020q,knys2019q,lqzg2021q,lll2020q,llll2020q,qpwy2019q,vkjk2021q}, more recent semi-local \mlps{}~\cite{gsvd2017q,sktm2017q,sstm2018q,um2019q,kgg2020q,kbg2021q,ucsm2021q,bmsk2022q,bkoc2022q,bbkc2022a,blcd2022q} have not yet been used for this purpose, partially because a heat flux formulation that incorporates message-passing mechanisms was not available.

\newthought{In this chapter, we fill that gap} and explain how to implement the \gk{} approach for such \mlps{} using \ad{}.

Since \ad relies on explicitly tracing the \scare{forward} computation, an understanding of how atomic potential energies are computed in practical implementations of such \mlps is required. \Cref{ch:glps} is dedicated to this task: It introduces the notion of \glps, which take a graph of \mic atom-pair vectors as inputs, encoding atomic neighbourhoods, and compute atomic potential energy contributions either based on neighbourhoods alone (local models) or can iteratively take neighbours-of-neighbours into account (semi-local models). As preparation for the heat flux, the definition and computation of forces and stress is discussed, unifying previous formulations for \ffs and \mlps.

Having established this foundation, we are then in a position to consider the computation of the heat flux in \cref{ch:gk-hf}.
As starting point, we first re-derive a general form of the heat flux, originally due to Hardy~\cite{h1963t}, to explicitly account for periodicity.
Then, strategies for efficiently implementing the heat flux with \ad are discussed, and the \scare{unfolded} heat flux is introduced, which is applicable to \glps and scales linearly with system size.

\subsection{Related publications}
Results of the presented work have been submitted for publication as:
\\\\
\PaperHF
\\\\
\PaperGLP
\\\\
In particular, the heat flux for semi-local \mlps was introduced in \cite{lksr2023a}, results for stress and \glps appear in \cite{lfk2023a}.

\subsection{Data and Code Availability}
The \glpc package uses \jax to implement the forces and stress from \cref{ch:glps}, and the heat flux formulations presented in \cref{ch:gk-hf}. It is available at \nicelink{https://github.com/sirmarcel/glp}. A \pytorch implementation of different heat flux formulations for \schnetpack is available at \nicelink{https://github.com/sirmarcel/gknet-archive}.
\\\\
Data and code related to \cref{tab:glps_lj_stress_error_32,tab:glps_snse_stress_error_32} can be found at\\
\nicedoi{10.5281/zenodo.7852530}.
